\documentclass[]{beamer}
\usepackage{pgfpages}

% \setbeameroption{show notes on second screen}

% Include any extra LaTeX packages required
\usepackage[
  backend=biber,
  style=alphabetic,
  citestyle=authoryear
]{biblatex}

\addbibresource{./library.bib}

%% Packages
  \usepackage{listings}
  \usepackage{multicol}
  \usepackage{booktabs}
  \usepackage{xcolor}
  \usepackage{times}
  \usepackage{tikz}
  \usepackage{amsmath}
  \usepackage{verbatim}
  \usetikzlibrary{arrows.meta,shapes,calc}

\usetheme{metropolis}

%% Settings and new commands
  \def\mAlertSpace{\vspace{0.5em}}
  \def\mOrangeItem{\item[\textcolor{mLightBrown}{\textbullet}]}
  % 36616a
  \definecolor{feldgrau}{rgb}{0.3, 0.36, 0.33}
  \newcommand{\mSlideTitle}{{{\color{gray}\secname}} \# \subsecname \hfill {\scriptsize$|$ UNED}}
  \newcommand{\mShortTitle}{\secname \hfill {\scriptsize$|$ UNED}}

  \tikzstyle{every picture}+=[remember picture]
  \everymath{\displaystyle}
  \input{json_listing.tex}

  \tikzset{
    invisible/.style={opacity=0},
    visible on/.style={alt={#1{}{invisible}}},
    alt/.code args={<#1>#2#3}{%
      \alt<#1>{\pgfkeysalso{#2}}{\pgfkeysalso{#3}} % \pgfkeysalso doesn't change the path
    },
  }

%% Preamble
  \title{Cross-lingual Training for\\Multiple-Choice Question Answering}
  \author{\textbf{Guillermo Echegoyen Blanco} \\ \'Alvaro Rodrigo \\ Anselmo Pe\~nas \\ 
    \{gblanco, alvarory, anselmo\} \textbf{at} lsi.uned.es \\ \\
    NLP \& IR Group \\
    National Distance Education University (UNED)}
    \titlegraphic{\hfill\includegraphics[height=0.75cm]{./figures/UNED.jpeg}
  }
  \date{}
  \addtobeamertemplate{frametitle}{}{%
    \begin{tikzpicture}[remember picture,overlay]
      \node[anchor=north east,yshift=-8pt,xshift=-8pt] at (current page.north east) {{\scriptsize$|$ UNED}};
    \end{tikzpicture}
  }


\begin{document}

\maketitle

\begin{frame}{Overview}
  \setbeamertemplate{section in toc}[sections numbered]
  \tableofcontents[hideallsubsections]
\end{frame}

\section{Introduction}
  \begin{frame}{\secname}
    \begin{alertblock}{Multiple-Choice Question Answering}
      \mAlertSpace
      \textbf{Def:} Given a supporting text, a question and a set of possible answers, choose the correct one. 
      \mAlertSpace
            \metroset{block=fill}
      \begin{exampleblock}{Example}
        \textbf{Evidence:} $\ldots$ Many people optimistically thought industry awards for better equipment would stimulate the production of quieter appliances. It was even suggested that noise from building sites could be alleviated $\ldots$ \\
        \textbf{Question:} \textit{What was the author’s attitude towards the industry awards for quieter?} \\
        \textbf{Options:} A. {\color{mLightGreen}suspicious} C. enthusiastic D. indifferent
      \end{alertblock}


    \end{alertblock}
  \end{frame}
  \begin{frame}{\secname}
    \begin{alertblock}{Multiple-Choice Question Answering}
      \mAlertSpace
      \begin{itemize}
        \item Measure reading comprehension in humans.
        \item Collections are usually extracted from exams for humans.
        \item Many real world exams are private.
        \item The majority of dataset are in English.
    \end{itemize}
    \end{alertblock}
  \end{frame}

  \note[itemize]{%
    \item Private exams are not suitable for research
  }

  \begin{frame}{\secname}
    \begin{alertblock}{Motivation}
      \begin{itemize}
        \item Scarce non-English datasets.
        \item Non-English datasets are usually small.
      \end{itemize}
    \end{alertblock}
  \end{frame}
  
  \note[itemize]{%
    \item Not enough data to train in non-English collections
  }

  \begin{frame}{\secname}
    \begin{alertblock}{Research Questions}
      \begin{itemize}
        \item How to zero-shot transfer from a big MC-QA collection to another one?
        \item Can we zero-shot transfer to another collection in a different language?
        \item Harder exams for humans are so for machines too?
      \end{itemize}
    \end{alertblock}
  \end{frame}

\section{Problem Statement}
  \begin{frame}{\secname}
    \begin{alertblock}{Datasets}
      \mAlertSpace
      \mAlertSpace
\mAlertSpace
\begin{columns}[T, onlytextwidth]
  \column{0.45\textwidth}
  RACE \\(\cite{Lai2017})
  \mAlertSpace
  \hline
  \mAlertSpace
    \begin{itemize}
      \item Chinese schools exams
      \item $>$ 97K Questions
      \item English (monolingual)
    \end{itemize}
  \column{0.5\textwidth}
  Entrance Exams \\
  (\cite{rodrigo_systems_2018})
  \mAlertSpace
  \hline
  \mAlertSpace
    \begin{itemize}
      \item University access in Japan
      \item $\approx$ 200 Questions
      \item 6 languages (multilingual)
    \end{itemize}
\end{columns}

    \end{alertblock}
  \end{frame}

  \note[itemize]{%
    \item RACE:
      \begin{enumerate}
        \item Divided into two collections: middle and high school.
      \end{enumerate}
    \item EE:
      \begin{enumerate}
        \item $\approx$ 500 times smaller than RACE. 
        \item Not suitable for fine tuning.
        \item Translated to Spanish, Italian, French, Russian, and (just one edition) German.
        \item Exams from three different years
      \end{enumerate}
    \item RACE are pre University exams, EE are exams at University level.
  }
  \begin{frame}{\secname}
    \begin{alertblock}{Models}
      \mAlertSpace
      \begin{itemize}
        \item BERT-base
        \item Multi BERT-base
      \end{itemize}
    \end{alertblock}
    \begin{alertblock}{Baselines}
      \mAlertSpace
      \begin{itemize}
        \item Random
        \item Longest answer (\cite{Rogers2020quail})
      \end{itemize}
    \end{alertblock}
  \end{frame}

\section{Experiments}
  \begin{frame}{\secname}
    \begin{alertblock}{Method}
      \begin{itemize}
        \item No hyper-parameters search.
        \item Fine-tune each model over RACE.
        \item Test each model over RACE.
        \item Test each model over Entrance Exams in all languages and all years
      \end{itemize}
    \end{alertblock}
  \end{frame}

\section{Results}
  \begin{frame}{\secname}
    \hspace{-1cm}
    \begin{tikzpicture}
  \node (table) {\input{./tables/results.tex}};
  \node (rect_res_1) [visible on=<2>] at (0,1.6) {
  \tikz{ \draw [mLightBrown,ultra thick,rounded corners]
    ($(table.south west) !.9! (table.north west)$)
    rectangle
    ($(table.south east) !.7! (table.north east)$);
  }};
  \node (res_text_1) [visible on=<2>] at (-2,-4) { \textbf{Difficulty affects machines too} };
  \path [-{Latex[width=2mm]}, visible on=<2>] (res_text_1.west) edge [bend left=25] (rect_res_1.west);
  \node (rect_res_2) [visible on=<3>] at (0,1.4) {
  \tikz{ \draw [mLightBrown,ultra thick,rounded corners]
    ($(table.south west) !.9! (table.north west)$)
    rectangle
    ($(table.south east) !.8! (table.north east)$);
  }};
  \node (rect_res_3)  [visible on=<3>] at (0,0.3) {
  \tikz{ \draw [mLightBrown,ultra thick,rounded corners]
    ($(table.south west) !.9! (table.north west)$)
    rectangle
    ($(table.south east) !.8! (table.north east)$);
  }};
  \node (res_text_2) [anchor=west, visible on=<3>] at (-5,-4) { \textbf{Pre-university graded exams results are comparable} };
  \path [-{Latex[width=2mm]}, visible on=<3>] (res_text_2.west) edge [bend left=25] (rect_res_2.west);
  \path [-{Latex[width=2mm]}, visible on=<3>] (res_text_2.west) edge [bend left=15] (rect_res_3.west);
  \node (rect_res_4)  [visible on=<4>] at (0,0.3) {
  \tikz{ \draw [mLightBrown,ultra thick,rounded corners]
    ($(table.south west) !.9! (table.north west)$)
    rectangle
    ($(table.south east) !.8! (table.north east)$);
  }};
  \node (res_text_4) [visible on=<4>] at (-2,-4) { \textbf{Best results are always in english} };
  \path [-{Latex[width=2mm]}, visible on=<4>] (res_text_4.west) edge [bend left=25] (rect_res_4.west);
\end{tikzpicture}

    \mAlertSpace
  \end{frame}

  \note[itemize]{%
    \item Multi-BERT performs better in all scenarios
    \item Russian is specially difficult as language with very different semantics are not well understood nor tokenized by the model.
    \item ** German only available for one year.
    \item ToDo := Information about previous results
    \item SOTA on Entrance Exams in several languages: Spanish, Italian and German. 
  }

\section{Conclusions \& Future Work}
  \begin{frame}{\secname}
    \begin{alertblock}{Conclusions}
      \begin{itemize}
        \item Performance holds across different tasks.
        \item Performance holds across languages in multilingual models.
        \item Performance drops with difficulty for humans.
      \end{itemize}
    \end{alertblock}
    \begin{alertblock}{Future Work}
      \begin{itemize}
        \item Transfer knowledge learnt in one language to another one.
      \end{itemize}
    \end{alertblock}
  \end{frame}

  \note[itemize]{%
    \item When exams are more difficult for humans, they are so for machines (Mid $<$ High $<$ EE)
    \item FW: specially when languages are low-resourced.
  }

\begin{frame}[standout]
  Thank you!\\
  Questions?
\end{frame}

\section{References}
  \begin{frame}[allowframebreaks]{References}
    \printbibliography%
  \end{frame}

\end{document}
